\documentclass[11pt, twoside, BCOR=16mm, a4paper, DIV=15, numbers=noenddot]{scrartcl}
\usepackage[utf8]{inputenc}
\usepackage{url}

% Mathematical Symbols/Miscellaneous Mathematical Symbols-A
%
\DeclareUnicodeCharacter{27E8}{\langle}%           ⟨  MATHEMATICAL LEFT ANGLE BRACKET
\DeclareUnicodeCharacter{27E9}{\rangle}%           ⟩  MATHEMATICAL RIGHT ANGLE BRACKET


\usepackage{amsmath}
\usepackage{graphicx}
\usepackage{xcolor}
\usepackage{rail}
\usepackage{listings}
\usepackage{tabularx}
\usepackage{multicol}

\usepackage[ngerman]{babel}
\usepackage{lastpage}

\usepackage{tikz}
\usetikzlibrary{arrows,trees,shapes.misc}

\tikzset{ level distance=2em, sibling distance=4em}

\usepackage{enumitem}
\newlist{teilaufg}{enumerate}{1}
\setlist{nosep}
\setlist[teilaufg]{label=(\alph*)}

\usepackage[T1]{fontenc}
\usepackage[bitstream-charter,cal=cmcal]{mathdesign}
\setkomafont{sectioning}{\normalcolor\bfseries}
\usepackage[scaled]{beramono}

\usepackage[thmmarks,amsmath]{ntheorem}
\theoremstyle{break}
\theorembodyfont{\upshape}
\theoremheaderfont{\bfseries}
\newtheorem{ex}{Übung}
\newtheorem{zu}{Zusatzaufgabe}

\renewcommand{\paragraph}[1]{\vspace{0.3cm}\noindent{\bf #1}}

\newcommand{\Nat}{\mathbb{N}}

\newcommand{\lam}[1]{\ensuremath{\langle #1 \rangle}}
\newcommand{\ite}{\lam{\mathrm{ite}}}
\newcommand{\iszero}{\lam{\mathrm{iszero}}}
\renewcommand{\succ}{\lam{\mathrm{succ}}}
\renewcommand{\mod}{\lam{\mathrm{mod}}}
\newcommand{\pred}{\lam{\mathrm{pred}}}
\newcommand{\true}{\lam{\mathrm{true}}}
\newcommand{\false}{\lam{\mathrm{false}}}
\newcommand{\add}{\lam{\mathrm{add}}}
\newcommand{\mult}{\lam{\mathrm{mult}}}
\newcommand{\sub}{\lam{\mathrm{sub}}}
\renewcommand{\epsilon}{\varepsilon}

\newcommand{\eor}{\mathop{|}}

\usepackage[footsepline,plainfootsepline]{scrpage2}
\clearscrheadfoot
\renewcommand{\headfont}{\normalfont}
\ofoot[\vspace*{-1mm}\pagemark\slash{}\pageref{LastPage}]{\vspace*{-1mm}\pagemark\slash{}\pageref{LastPage}}
\pagestyle{scrplain}

\newcommand\starred{\ensuremath{\star}}

\setlength{\theorempreskipamount}{10pt}
\setlength{\theorempostskipamount}{15pt}


\lstdefinelanguage{Haskell2010}%
{ sensitive%
  , morecomment=[l]--%
  , morecomment=[n]{\{-}{-\}}%
  , morestring=[b]"%
  % , morestring=[b]'% this would collide with identifiers containing '
  , alsodigit={'.}%
  , alsoother={@\$}%
  % http://www.haskell.org/onlinereport/haskell2010/haskellch2.html#x7-180002.4
  , morekeywords={case, class, data, default, deriving, do, else, foreign, if, import, in, infix, infixl, infixr, instance, let, module, newtype, of, then, type, where}%
}

\tikzset{% für den Hoare-Kalkül
  HoareNode/.style={draw, inner sep=5pt,rounded corners=8pt},
  HoareKante/.style={above left,inner sep=1.5pt},
  HoareTree/.style={%
    grow'=up,%
    line width=1pt,%
    every child node/.style={HoareNode},
    sibling distance=1cm, level distance=2cm,%
    edge from parent/.style={draw,latex'-},edge from parent path={(\tikzparentnode) -- (\tikzchildnode)}%
  },
  HoareImpl/.style={rounded corners=0pt},
}

\lstdefinestyle{oholzer}
{ basicstyle=\ttfamily
  , identifierstyle=
  , stringstyle=
  , keywordstyle=\bfseries
  , captionpos=b
  , columns=fullflexible
  , keepspaces=true
  , sensitive=true
  , showstringspaces=false
  , numbers=none
  , firstnumber=auto  % \thelstnumber
  , numberblanklines=true
  , showlines=false  % print empty lines at the end of listings
  , showspaces=false
  , showtabs=false
  %	, numbersep=1.1em
  %	, framerule=0.5pt
  %	, rulesep=-2.4em
  , tabsize=2
  , xleftmargin=3\parindent
  , language=Haskell2010}

\lstset{style=oholzer}


\makeatletter
\def\hrulefill{\leavevmode\leaders \hrule height \rulethickness \hfill\kern\z@}
\makeatletter

\begin{document}

\vspace*{-5mm}\hspace*{-4mm}\includegraphics[height = 8mm]{TU_Logo_SW}
\quad
\begin{minipage}[b]{12.3cm}
  % \begin{tabular}{p\textwidth}
  \hrulefill\\
  \scriptsize
  % \vspace*{-2mm}
  {\scriptsize{Fakult\"at Informatik} {$\bullet$}
    Institut f\"ur Theoretische Informatik {$\bullet$}
    Lehrstuhl Grundlagen der Programmierung }
  Prof. Dr. H.\,Vogler / Dipl.-Inf. J.\,Osterholzer \hfill\url{http://www.orchid.inf.tu-dresden.de}\\[-2.1mm]
  \mbox{}\hrulefill
  % \end{tabular}
\end{minipage}

\begin{center}
  \huge Programmierung\\
  \Large Lösungen zum 13. Übungsblatt\\[1mm]
  \small Zeitraum: 11. -- 15. Juli 2016\\[3mm]
\end{center}

\begin{ex}
  \textbf{Induktionsanfang }(IA) mit \texttt{r = []}:
Sei \texttt{x :: Int}. Dann gilt:
\begin{align*}
  \texttt{sumTree (toTree [] x)}&\stackrel{12}{=}\texttt{sumTree (Leaf x)}\\
  &\stackrel{4}{=}\texttt{x}\\
  &=\texttt{x * 1}\\
  &\stackrel{8}{=}\texttt{x * prod []}\\
\end{align*}

\noindent\textbf{Induktionsvoraussetzung }(IV):
Sei \texttt{r :: [Int]} beliebig aber fest so dass
\begin{center}
  \texttt{sumTree (toTree r x) = x * prod r}\;
\end{center}
für jedes \texttt{x :: Int}.
 
\noindent\textbf{Induktionsschritt }(IS):
Sei \texttt{a :: Int} beliebig.
\begin{align*}
  \texttt{sumTree(toTree (a : r) x) }&\stackrel{13}{=}\texttt{sumTree(Node (toTree r (2 * a * x)) (toTree r (-a * x)))}\\
  &\stackrel{5}{=}\texttt{sumTree (toTree r (2 * a * x)) + sumTree (toTree r (-a * x))}\\
  &\stackrel{IV}{=}\texttt{2 * a * x * prod r + (-a * x) * prod r}\\
  &=\texttt{a * x * prod r}\\
  &=\texttt{x * (a * prod r)}\\
  &\stackrel{9}{=}\texttt{x * prod (a : r)}\\
\end{align*}

\end{ex}

\begin{ex}
  \begin{teilaufg}
  \item~
      
    $(y (\lambda x.y) x) ((\lambda x\underbrace{y.x (\lambda z.z) y}_{GV=\{y,z\}}) \underbrace{(\lambda x.y)}_{FV=\{y\}})$ \\
    $\Rightarrow_\alpha (y (\lambda x.y) x) ((\lambda x\underbrace{y_1.x (\lambda z.z) y_1}_{GV=\{y_1,z\}}) \underbrace{(\lambda x.y)}_{FV=\{y\}})$ \\
    $\Rightarrow_\beta (y (\lambda x.y) x) (\lambda y_1.(\lambda x.\underbrace{y}_{GV=\emptyset}) \underbrace{(\lambda z.z)}_{FV=\emptyset} y_1)$ \\
    $\Rightarrow_\beta (y (\lambda x.y) x) (\lambda y_1.y y_1)$
%
    \item~
    \begin{align*}
    ⟨G⟩ = & (\lambda fxy. ⟨\mathit{ite}⟩ (⟨\mathit{iszero}⟩ x) \; y \\
    & \quad (⟨\mathit{ite}⟩ (⟨\mathit{iszero}⟩ (⟨\mathit{mod}⟩ x ⟨2⟩)) \\
    & \qquad (⟨\mathit{add}⟩ ⟨3⟩ (f (⟨\mathit{pred}⟩ x) (⟨\mathit{succ}⟩ (⟨\mathit{succ}⟩ y)))) \\
    & \qquad (⟨\mathit{mult}⟩ ⟨2⟩ (f (⟨\mathit{pred}⟩ x) (⟨\mathit{succ}⟩ y)))))
    \end{align*}
%
    \item~
  
      Nebenrechnung: $⟨Y⟩⟨F⟩ = (\lambda h. \; (\lambda y.h\;(yy))\; (\lambda y.h\; (yy))) ⟨F⟩ \\
    \Rightarrow_\beta \underbrace{(\lambda y.⟨F⟩(yy))\; (\lambda y.⟨F⟩(yy))}_{=⟨Y_F⟩}
    \Rightarrow_\beta \; ⟨F⟩ ⟨Y_F⟩$\\[\baselineskip]
%
    $⟨Y⟩⟨F⟩⟨1⟩⟨5⟩\\
    \\
    \Rightarrow^{*} \; ⟨F⟩ ⟨Y_F⟩ ⟨1⟩⟨5⟩\\
    \Rightarrow^{*} \; ⟨\mathit{ite}⟩ (\underbrace{⟨\mathit{iszero}⟩ ⟨1⟩}_{\Rightarrow^{*} \; ⟨\mathit{false}⟩}) (⟨\mathit{succ}⟩ (⟨\mathit{succ}⟩ ⟨5⟩))\\
    \mbox{}\qquad\qquad (⟨\mathit{add}⟩ ⟨5⟩ ( \underbrace{⟨Y_F⟩}_{\Rightarrow^{*} \; ⟨F⟩ ⟨Y_F⟩} (\underbrace{⟨\mathit{pred}⟩ ⟨1⟩}_{\Rightarrow^{*} \; ⟨0⟩}) (\underbrace{⟨\mathit{succ}⟩ ⟨1⟩}_{\Rightarrow^{*} \; ⟨2⟩})\\[\baselineskip]
%
    \Rightarrow^{*} \; ⟨\mathit{add}⟩ ⟨5⟩ (⟨F⟩ ⟨Y_F⟩ ⟨0⟩ ⟨2⟩)\\
    \Rightarrow^{*} \; ⟨\mathit{add}⟩ ⟨5⟩ (⟨\mathit{ite}⟩ (\underbrace{⟨\mathit{iszero}⟩ ⟨0⟩}_{\Rightarrow^{*} \; ⟨\mathit{true}⟩}) (\underbrace{⟨\mathit{succ}⟩ (\underbrace{⟨\mathit{succ}⟩ ⟨2⟩}_{\Rightarrow^{*} \; ⟨3⟩})}_{\Rightarrow^{*} \; ⟨4⟩}) \\
    \mbox{}\qquad\qquad(⟨\mathit{add}⟩ ⟨2⟩ (⟨Y_F⟩ (⟨\mathit{pred}⟩ ⟨0⟩) (⟨\mathit{succ}⟩ ⟨2⟩))))\\[\baselineskip]
%
    \Rightarrow^{*} \; ⟨\mathit{add}⟩ ⟨5⟩ ⟨4⟩\\[\baselineskip]
%
    \Rightarrow^{*} \; ⟨9⟩$
\end{teilaufg}
\end{ex}

\begin{ex}
  \begin{teilaufg}
\item
\setlength\abovedisplayskip{-\baselineskip}
\begin{align*}
	\mathit{tab}_{g + \mathit{lDecl}} = [
&
	f / (\textnormal{proc}, 1),
	g / (\textnormal{proc}, 2),
	x / (\textnormal{var}, \textnormal{global}, 1),
	y / (\textnormal{var}, \textnormal{global}, 2),
\\&
	b / (\textnormal{var-ref}, -2),
	a / (\textnormal{var}, \textnormal{lokal}, 1) ]
\end{align*}
AM$_1$-Code:
\begin{lstlisting}[numbers=none]
2.1.1:  LOAD (global,1); LIT 0; GT;
        JMC 2.1.2;
        LOAD (global,1); LOAD (lokal,1); SUB; STORE (global,1);
        LOAD (global,2); PUSH;
        LOADA (lokal,1); PUSH;
        CALL 1;
        JMP 2.1.1;
2.1.2:  LOAD(lokal,-2); PUSH; CALL 2; 
\end{lstlisting}

\item
\begin{tabular}[t]{c|r|l|c|c|c}
BZ& DK&LK&REF&Inp&Out \\ \hline \hline
$13$&$\varepsilon$&$0:3:0:9:7$&$3$&$\varepsilon$&$\varepsilon$\\ \hline
$14$&$7$&$0:3:0:9:7$&$3$&$\varepsilon$&$\varepsilon$\\ \hline
$15$&$\varepsilon$&$0:3:0:9:7:7$&$3$&$\varepsilon$&$\varepsilon$\\ \hline
$16$&$4$&$0:3:0:9:7:7$&$3$&$\varepsilon$&$\varepsilon$\\ \hline
$17$&$\varepsilon$&$0:3:0:9:7:7:4$&$3$&$\varepsilon$&$\varepsilon$\\ \hline
$4$&$\varepsilon$&$0:3:0:9:7:7:4:18:3$&$9$&$\varepsilon$&$\varepsilon$\\ \hline
$5$&$\varepsilon$&$0:3:0:9:7:7:4:18:3$&$9$&$\varepsilon$&$\varepsilon$\\ \hline
$6$&$9$&$0:3:0:9:7:7:4:18:3$&$9$&$\varepsilon$&$\varepsilon$\\ \hline
$7$&$7:9$&$0:3:0:9:7:7:4:18:3$&$9$&$\varepsilon$&$\varepsilon$\\ \hline
$8$&$16$&$0:3:0:9:7:7:4:18:3$&$9$&$\varepsilon$&$\varepsilon$\\ \hline
$9$&$\varepsilon$&$16:3:0:9:7:7:4:18:3$&$9$&$\varepsilon$&$\varepsilon$\\ \hline
$18$&$\varepsilon$&$16:3:0:9:7$&$3$&$\varepsilon$&$\varepsilon$\\ \hline
$19$&$\varepsilon$&$16:3:0:9:7$&$3$&$\varepsilon$&$16$\\ \hline
$3$&$\varepsilon$&$16$&$0$&$\varepsilon$&$16$\\ \hline
$0$&$\varepsilon$&$16$&$0$&$\varepsilon$&$16$\\ \hline
\end{tabular}
\end{teilaufg}
\end{ex}

\begin{ex}
  \begin{teilaufg}
\item $SI = (z=(x-x1)\cdot 3y)\wedge(x1\geq0)$
%
\item 
\begin{tabular}[t]{@{}l}
$A=C=SI$\\
$B=SI\wedge\neg\pi=(z=(x-x1)\cdot 3y)\wedge(x1\geq0)\wedge(x1\leq0)$\\
$D=SI\wedge\pi=(z=(x-x1)\cdot 3y)\wedge(x1\geq0)\wedge(x1>0)$
\end{tabular}
\end{teilaufg}
\end{ex}


\begin{zu}
  ~
  \begin{lstlisting}[language=Haskell2010]
-- (a)
expo :: Int -> Int -> Int
expo 0 0 = error "0^0 ist nicht definiert!"
expo x 1 = x
expo _ 0 = 1
expo x n = x * expo x (n - 1)

-- (b)
data Tree a = Branch (Tree a) (Tree a) | Leaf a deriving Show

check :: Tree a -> Int -> Bool
check (Leaf _)     k = k == 0
check (Branch l r) k = check l (k - 1) || check r (k - 1)

-- (c)
test :: [Int] -> Bool
test []       = True
test (z : zs) = notElem z zs && test zs
  where
    notElem :: Int -> [Int] -> Bool
    notElem _ []       = True
    notElem x (y : ys) = x /= y && notElem x ys
\end{lstlisting}
\emph{Hinweis:} Die Funktion \lstinline!notElem! ist normalerweise im Modul \lstinline!Prelude! bereits mit folgendem Typ vordefiniert: \lstinline!notElem :: Eq a => a -> [a] -> Bool!.

\end{zu}

% Unifikation
\newcommand{\uniftuple}[2]{\begin{pmatrix}#1\\ #2\end{pmatrix}}
\newcommand*{\unifArr}[1]{\stackrel{\smash{\text{#1}}}{\Longrightarrow}}
\newcommand*{\unifArrD}{\unifArr{Dek.}}% Dekomposition
\newcommand*{\unifArrE}{\unifArr{El.}}% Elimination
\newcommand*{\unifArrP}{\unifArr{Vert.}}% Vertauschung (Permutation)
\newcommand*{\unifArrS}{\unifArr{Sub.}}% Substitution


\begin{zu}
  \begin{teilaufg}
\item\mbox{}\vspace{-2em}
	\begin{align*}
		& \left\{\uniftuple{\underline{\sigma}(\tau( x_4, x_2 ),\sigma(\gamma( x_1 ), x_1 ))}{\underline{\sigma}(x_1,\sigma( x_3 , \tau(\alpha, x_2 )))}\right\}
	\\
		\unifArrD {}& \left\{ \uniftuple{\tau( x_4, x_2 )}{ x_1}, \uniftuple{\underline{\sigma}(\gamma( x_1 ), x_1 )}{\underline{\sigma}( x_3 , \tau(\alpha, x_2))} \right\}
	\\
		\unifArrD {}& \left\{ \uniftuple{\tau( x_4, x_2 )}{ \underline{x_1}}, \uniftuple{\gamma( \underline{x_1})}{x_3 }, \underline{\uniftuple{x_1}{\tau(\alpha, x_2)} } \right\}
	\\
		\unifArrS {}& \left\{ \uniftuple{\underline{\tau}( x_4, x_2 )}{\underline{\tau}(\alpha, x_2)}, \uniftuple{\gamma( \tau(\alpha, x_2))}{x_3 }, \uniftuple{x_1}{\tau(\alpha, x_2)}\right\}
	\\
		\unifArrD {}& \left\{ \uniftuple{x_4}{ \alpha}, \underline{\uniftuple{x_2}{ x_2}}, \uniftuple{\gamma( \tau(\alpha, x_2))}{x_3 }, \uniftuple{x_1}{\tau(\alpha, x_2)}\right\}
	\\
		\unifArrE {}& \left\{ \uniftuple{x_4}{ \alpha}, \underline{\uniftuple{\gamma( \tau(\alpha, x_2))}{x_3 }}, \uniftuple{x_1}{\tau(\alpha, x_2)}\right\}
	\\
		\unifArrP {}& \left\{ \uniftuple{x_4}{ \alpha}, \uniftuple{x3}{ \gamma( \tau(\alpha, x_2)) }, \uniftuple{x_1}{\tau(\alpha, x_2)}\right\}
	\end{align*}
	allgemeinster Unifikator: $x_1\mapsto\tau(\alpha, x_2),\ x_2\mapsto x_2,\ x_3\mapsto \gamma( \tau(\alpha, x_2)),\ x_4\mapsto \alpha$
\item
	\(t_1 = \gamma(x_1)\), \(t_2 = \gamma(\gamma(x_1))\)
\end{teilaufg}
\end{zu}

\end{document}

%%% Local Variables:
%%% mode: latex
%%% TeX-master: t
%%% End:
